\documentclass[conference]{IEEEtran}
\usepackage[utf8]{inputenc}
\usepackage[T1]{fontenc}
\usepackage{amsmath}
\usepackage{amssymb}
\usepackage{graphicx}
\usepackage{float}
\usepackage{url}
\usepackage{multirow}
\usepackage{subcaption}
\usepackage{float}


\begin{document}

\title{
    CMP Performance Analysis with gem5 }

\author{
    \begin{tabular}{@{}c c c@{}}
        \begin{minipage}[t]{0.29\textwidth}
            \centering
            \textbf{Javier Andres Tarazona Jimenez}\\
            \textit{Ingénieur Degree Programme STIC}\\
            \textit{ENSTA Paris}\\
            Paris, France\\
            \url{javier-andres.tarazona@ensta-paris.fr}
        \end{minipage}
        &
        \begin{minipage}[t]{0.29\textwidth}
            \centering
            \textbf{Jair Anderson Vasquez Torres}\\
            \textit{Ingénieur Degree Programme STIC}\\
            \textit{ENSTA Paris}\\
            Paris, France\\
            \url{jair-anderson.vasquez@ensta-paris.fr}
        \end{minipage}
        &
        \begin{minipage}[t]{0.29\textwidth}
            \centering
            \textbf{Maeva Noukoua}\\
            \textit{Ingénieur Degree Programme STIC}\\
            \textit{ENSTA Paris}\\
            Paris, France\\
            \url{maeva-sandy.noukoua@ensta-paris.fr}
        \end{minipage}
        \\[1.2em]
        \multicolumn{3}{c}{
            \begin{minipage}[t]{0.29\textwidth}
                \centering
                \textbf{Carlos Adrian Meneses Gamboa}\\
                \textit{Ingénieur Degree Programme STIC}\\
                \textit{ENSTA Paris}\\
                Paris, France\\
                \url{carlos-adrian.meneses@ensta.fr}
            \end{minipage}
        }
    \end{tabular}
}

\maketitle

\begin{abstract}
Ce rapport présente une étude de performances d'un processeur multi-c\oe{}urs (CMP) simulé avec gem5 sur un benchmark OpenMP de multiplication de matrices.
Nous analysons les effets de cohérence de cache et les paramètres par défaut du modèle CPU et de la hiérarchie mémoire.
Nous mesurons ensuite, sur des c\oe{}urs Cortex-A7 in-order puis Cortex-A15 out-of-order, l'évolution des cycles, du speedup et de l'IPC en fonction du nombre de threads, et pour A15 en fonction de la largeur superscalaire.
Les résultats mettent en évidence un gain important mais sous-linéaire, limité par les synchronisations OpenMP et la pression mémoire/cohérence, ainsi que des rendements décroissants lorsque la largeur augmente.
Enfin, nous discutons des cas pouvant conduire à un speedup supra-linéaire lorsque l'ensemble de travail tient dans les caches privés.
\end{abstract}

\begin{IEEEkeywords}
gem5, architecture multi-c\oe{}ur (CMP), OpenMP, cohérence de cache, speedup, IPC, Cortex-A7, Cortex-A15, superscalaire.
\end{IEEEkeywords}


\section*{Configuration Expérimentale}

Pour l'analyse des performances, les environnements suivants ont été utilisés :
\begin{itemize}
    \item \textbf{Configuration Séquentielle (TP4) :} Station de travail type PC portable (Intel Core i7 8750H, 16 Go RAM) sous environnement WSL2.
    \item \textbf{Configuration Parallèle (TP5) :} Serveur de calcul de l'école (Intel Xeon E5-2640 v2 @ 2.00GHz) via SSH.
\end{itemize}

\section{Analyse théorique de cohérence de cache}
\raggedbottom

\subsection{Q1 --- Cohérence de cache et accès mémoire de \texttt{test\_omp}}

\paragraph{Contexte.}
On considère une architecture CMP (plusieurs cœurs) avec un bus partagé et des caches privés par cœur (L1).
Le programme \texttt{test\_omp} réalise une multiplication de matrices \(C = A \times B\) en parallèle via OpenMP.
Dans ce TP, le nombre de threads est couplé au nombre de cœurs : \(nthreads = ncores\).

\paragraph{Nature des accès.}
Dans une multiplication de matrices, \(A\) et \(B\) sont essentiellement \textbf{lus}, tandis que \(C\) est \textbf{écrite}.
Le parallélisme OpenMP répartit typiquement des lignes/blocs de \(C\) entre threads ; idéalement, chaque thread écrit dans une zone disjointe.

\paragraph{Cohérence (intuition type MESI).}
Avec des caches privés, une même ligne mémoire peut être présente dans plusieurs caches.
Un protocole de cohérence garantit qu'une écriture rend visibles les mises à jour :
\begin{itemize}
  \item \textbf{Read miss :} chargement d'une ligne via le bus (depuis mémoire/niveau inférieur), puis réutilisation locale si la ligne reste en cache.
  \item \textbf{Write miss :} obtention de l'exclusivité sur la ligne, impliquant l'\textbf{invalidation} des copies chez les autres cœurs.
  \item \textbf{Writes successifs :} une fois la ligne exclusive, les écritures suivantes sont locales jusqu'à éviction.
\end{itemize}

\paragraph{Effets attendus sur \(A\) et \(B\) (lectures partagées).}
Comme \(A\) et \(B\) sont read-only durant le calcul, les lignes peuvent être \emph{partagées} dans les caches (état \emph{Shared}).
Le trafic bus provient surtout des accès initiaux (miss de froid) et des évictions lorsque la capacité L1 est insuffisante.

\paragraph{Effets attendus sur \(C\) (écritures).}
Pour écrire \(C\), chaque cœur doit obtenir l'exclusivité sur les lignes correspondantes.
Si les zones écrites par thread sont disjointes et alignées, la cohérence génère peu d'interactions.
En revanche, le \textbf{false sharing} peut apparaître : deux threads écrivent des éléments différents mais situés sur la \emph{même ligne de cache},
ce qui provoque une alternance d'invalidations (« ping-pong ») et dégrade fortement le speedup.

\paragraph{Conséquence sur la scalabilité.}
On s'attend à :
\begin{itemize}
  \item un bon gain initial en augmentant \(ncores\) (parallélisme TLP) ;
  \item puis une saturation due à (i) contention mémoire/bus, (ii) surcoûts de cohérence (notamment via \(C\)), (iii) barrières OpenMP,
        et (iv) limites de capacité de cache (miss rate).
\end{itemize}

\section{Paramètres de l'architecture multicoeurs}

\subsection{Q2 --- Paramètres par défaut d'un CPU OoO (BaseO3CPU)}

\paragraph{Remarque.}
Le fichier \texttt{O3CPU.py} sélectionne la variante OoO selon l'ISA.
Les \textbf{paramètres par défaut} se trouvent dans \texttt{src/cpu/o3/BaseO3CPU.py}.

\paragraph{Paramètres OoO retenus (valeurs par défaut).}
Nous reportons ci-dessous des paramètres structurants pour un cœur superscalaire out-of-order
(largeurs du pipeline et taille des structures de renommage/exécution), ainsi que leur interprétation.

\begin{table}[htbp]
\centering
\scriptsize
\caption{Paramètres OoO par défaut (extraits de \texttt{BaseO3CPU.py}) et rôle.}
\label{tab:q2_o3_params}
\begin{tabular}{|l|c|p{4.8cm}|}
\hline
\textbf{Paramètre} & \textbf{Défaut} & \textbf{Rôle / effet attendu} \\
\hline
\texttt{fetchWidth} & 8 & Max instructions fetch/cycle : alimente le pipeline ; utile si le front-end est limitant. \\
\hline
\texttt{decodeWidth} & 8 & Max decode/cycle : limite la capacité de décodage avant renommage/dispatch. \\
\hline
\texttt{issueWidth} & 8 & Max instructions émises/cycle : augmente l'ILP exploitable si dépendances faibles. \\
\hline
\texttt{commitWidth} & 8 & Max instructions retirées/cycle : borne finale du débit d'instructions (retirement). \\
\hline
\texttt{numROBEntries} & 192 & Taille du ROB : fenêtre OoO plus large $\Rightarrow$ meilleure tolérance aux latences (mémoire/branches). \\
\hline
\end{tabular}
\end{table}

\paragraph{Lecture.}
Ces valeurs indiquent un cœur OoO « large » (largeurs à 8) avec une fenêtre OoO conséquente (ROB=192).
Dans le cadre du TP, augmenter la largeur (via \texttt{o3-width}) revient à contraindre/décontraindre ces largeurs effectives,
ce qui agit sur la capacité à exploiter l'ILP à l'intérieur de chaque thread.

\subsection{Q3 --- Valeurs par défaut des caches (Options.py)}

\paragraph{Valeurs par défaut.}
D'après \texttt{configs/common/Options.py}, les paramètres par défaut sont :
\begin{itemize}
  \item \(L1D = 64\ \mathrm{KiB}\), associativité \(2\)-ways ;
  \item \(L1I = 32\ \mathrm{KiB}\), associativité \(2\)-ways ;
  \item \(L2 = 2\ \mathrm{MiB}\), associativité \(8\)-ways ;
  \item taille de ligne = \(64\ \mathrm{B}\).
\end{itemize}

\begin{table}[htbp]
\centering
\scriptsize
\caption{Paramètres de caches par défaut (Options.py).}
\label{tab:q3_cache_defaults}
\begin{tabular}{|l|c|c|}
\hline
\textbf{Niveau} & \textbf{Taille} & \textbf{Associativité} \\
\hline
L1D & 64KiB & 2 \\
\hline
L1I & 32KiB & 2 \\
\hline
L2  & 2MiB  & 8 \\
\hline
Cache line & 64B & -- \\
\hline
\end{tabular}
\end{table}

\paragraph{Commentaire.}
Ces valeurs conditionnent le taux de miss et la pression mémoire.
En multicœur, une augmentation de \(ncores\) augmente le volume de requêtes concurrentes, ce qui rend plus visibles
la contention sur le bus/mémoire et les surcoûts de cohérence.


\section{Paramètres de l'architecture multicoeurs}

\subsection{Q2 --- Paramètres par défaut d'un CPU OoO (BaseO3CPU)}

\paragraph{Remarque.}
Le fichier \texttt{O3CPU.py} sélectionne la variante OoO selon l'ISA.
Les \textbf{paramètres par défaut} se trouvent dans \texttt{src/cpu/o3/BaseO3CPU.py}.

\paragraph{Paramètres OoO retenus (valeurs par défaut).}
Nous reportons ci-dessous des paramètres structurants pour un cœur superscalaire out-of-order
(largeurs du pipeline et taille des structures de renommage/exécution), ainsi que leur interprétation.

\begin{table}[htbp]
\centering
\scriptsize
\caption{Paramètres OoO par défaut (extraits de \texttt{BaseO3CPU.py}) et rôle.}
\label{tab:q2_o3_params}
\begin{tabular}{|l|c|p{4.8cm}|}
\hline
\textbf{Paramètre} & \textbf{Défaut} & \textbf{Rôle / effet attendu} \\
\hline
\texttt{fetchWidth} & 8 & Max instructions fetch/cycle : alimente le pipeline ; utile si le front-end est limitant. \\
\hline
\texttt{decodeWidth} & 8 & Max decode/cycle : limite la capacité de décodage avant renommage/dispatch. \\
\hline
\texttt{issueWidth} & 8 & Max instructions émises/cycle : augmente l'ILP exploitable si dépendances faibles. \\
\hline
\texttt{commitWidth} & 8 & Max instructions retirées/cycle : borne finale du débit d'instructions (retirement). \\
\hline
\texttt{numROBEntries} & 192 & Taille du ROB : fenêtre OoO plus large $\Rightarrow$ meilleure tolérance aux latences (mémoire/branches). \\
\hline
\end{tabular}
\end{table}

\paragraph{Lecture.}
Ces valeurs indiquent un cœur OoO « large » (largeurs à 8) avec une fenêtre OoO conséquente (ROB=192).
Dans le cadre du TP, augmenter la largeur (via \texttt{o3-width}) revient à contraindre/décontraindre ces largeurs effectives,
ce qui agit sur la capacité à exploiter l'ILP à l'intérieur de chaque thread.

\subsection{Q3 --- Valeurs par défaut des caches (Options.py)}

\paragraph{Valeurs par défaut.}
D'après \texttt{configs/common/Options.py}, les paramètres par défaut sont :
\begin{itemize}
  \item \(L1D = 64\ \mathrm{KiB}\), associativité \(2\)-ways ;
  \item \(L1I = 32\ \mathrm{KiB}\), associativité \(2\)-ways ;
  \item \(L2 = 2\ \mathrm{MiB}\), associativité \(8\)-ways ;
  \item taille de ligne = \(64\ \mathrm{B}\).
\end{itemize}

\begin{table}[htbp]
\centering
\scriptsize
\caption{Paramètres de caches par défaut (Options.py).}
\label{tab:q3_cache_defaults}
\begin{tabular}{|l|c|c|}
\hline
\textbf{Niveau} & \textbf{Taille} & \textbf{Associativité} \\
\hline
L1D & 64KiB & 2 \\
\hline
L1I & 32KiB & 2 \\
\hline
L2  & 2MiB  & 8 \\
\hline
Cache line & 64B & -- \\
\hline
\end{tabular}
\end{table}

\paragraph{Commentaire.}
Ces valeurs conditionnent le taux de miss et la pression mémoire.
En multicœur, une augmentation de \(ncores\) augmente le volume de requêtes concurrentes, ce qui rend plus visibles
la contention sur le bus/mémoire et les surcoûts de cohérence.


\input{sections/pt3_A7.tex}

\section{Architecture multicoeurs avec des processeurs superscalaires 
out-of-order (Cortex A15 - Q9-11)}
\raggedbottom

\subsection{Stratégie adoptée pour traiter la Q9}

Pour répondre à la Q9 (faire varier le nombre de threads et la largeur superscalaire, puis produire un graphe 3D des cycles), nous avons mis en place une chaîne reproductible en quatre scripts :

\begin{itemize}
  \item un script d'orchestration des simulations,
  \item un script gem5 SE adapté au modèle A15/o3,
  \item un script de post-traitement pour extraire les cycles et générer la visualisation,
  \item un script dédié à l'extraction/calcul de l'IPC.
\end{itemize}

Cette organisation permet de lancer une campagne complète, reprendre après erreur, tracer précisément chaque exécution, et générer automatiquement le CSV et la figure 3D demandés.

\subsection{Script \texttt{run\_q9\_a15.sh}: orchestration de la campagne}

\textbf{Ce qu'il fait :}
\begin{itemize}
  \item lance toutes les combinaisons \texttt{(size, width, threads)} pour Q9 ;
  \item crée un répertoire de sortie par combinaison ;
  \item enregistre l'état d'avancement dans \texttt{state.tsv} (\texttt{PENDING}, \texttt{DONE}, \texttt{FAILED}) ;
  \item permet la reprise automatique après interruption/échec ;
  \item journalise chaque run dans un fichier de log dédié.
\end{itemize}

\textbf{Comment il le fait :}
\begin{itemize}
  \item construit la commande gem5 avec \texttt{--cpu-type=detailed}, \texttt{--o3-width}, \texttt{--num-cpus}, et les arguments du benchmark ;
  \item exécute les runs séquentiellement et s'arrête au premier échec pour conserver un diagnostic clair ;
  \item relit \texttt{state.tsv} au redémarrage pour ignorer les cas déjà \texttt{DONE} ;
  \item supporte un mode de mitigation OpenMP (\texttt{--omp-active-wait}) via un fichier d'environnement passé à gem5.
\end{itemize}

\subsection{Script \texttt{se\_a15.py}: configuration gem5 pour A15/o3}

\textbf{Ce qu'il fait :}
\begin{itemize}
  \item instancie un système gem5 en mode syscall emulation (SE) ;
  \item configure des CPU de type \texttt{detailed} (modèle o3) ;
  \item applique la largeur superscalaire via \texttt{o3-width} ;
  \item exécute le binaire \texttt{test\_omp} avec les paramètres \texttt{threads} et \texttt{size}.
\end{itemize}

\textbf{Comment il le fait :}
\begin{itemize}
  \item s'appuie sur les options standard gem5 (\texttt{Options.addCommonOptions}, \texttt{Options.addSEOptions}) ;
  \item crée \texttt{num-cpus} cœurs simulés et fixe \texttt{issueWidth} pour chaque cœur ;
  \item configure hiérarchie mémoire/caches et lance la simulation avec \texttt{Simulation.run()} ;
  \item prend en charge un fichier \texttt{--env} pour injecter des variables OpenMP/libgomp si nécessaire.
\end{itemize}

\subsection{Script \texttt{plot\_q9\_cycles.py}: extraction et visualisation}

\textbf{Ce qu'il fait :}
\begin{itemize}
  \item lit \texttt{state.tsv} et sélectionne les runs \texttt{DONE} valides ;
  \item extrait les cycles à partir des \texttt{stats.txt} de gem5 ;
  \item génère un CSV consolidé ;
  \item produit le graphe 3D demandé (threads, largeur, cycles).
\end{itemize}

\textbf{Comment il le fait :}
\begin{itemize}
  \item vérifie les colonnes attendues de \texttt{state.tsv} et la présence des fichiers \texttt{stats.txt} ;
  \item récupère la métrique \texttt{system.cpu*.numCycles} et conserve la valeur maximale par run ;
  \item écrit \texttt{q9\_cycles.csv} puis trace \texttt{q9\_cycles\_3d.png} avec Matplotlib ;
  \item signale explicitement les combinaisons manquantes/invalides non incluses dans la figure.
\end{itemize}

\subsection{Script \texttt{extract\_q9\_ipc.py}: extraction de l'IPC}

\textbf{Ce qu'il fait :}
\begin{itemize}
  \item extrait \texttt{sim\_insts} et \texttt{numCycles} des runs \texttt{DONE} ;
  \item calcule l'IPC pour chaque configuration \((width,threads)\) ;
  \item exporte les résultats détaillés et les maxima.
\end{itemize}

\textbf{Comment il le fait :}
\begin{itemize}
  \item lit \texttt{state.tsv}, filtre les runs valides et ouvre chaque \texttt{stats.txt} ;
  \item utilise \(\max(\texttt{system.cpu*.numCycles})\) en multicœur, avec fallback \texttt{system.cpu.numCycles} en mono-cœur ;
  \item écrit \texttt{results/images/A15/q9\_ipc.csv} et \texttt{results/images/A15/q9\_ipc\_max.csv}.
\end{itemize}

\subsection{Expérimentation}

L'expérimentation a été lancée avec le script \texttt{run\_q9\_a15.sh} en conservant les valeurs par défaut du script :
\texttt{size=64}, \texttt{widths=\{2,4,8\}}, \texttt{threads} en puissances de 2, caches activés, et sorties dans \texttt{results/A15}.

Nous visions initialement une exploration jusqu'à 64 threads. En pratique, les exécutions à forte concurrence ont présenté des \texttt{SIGSEGV} de gem5 (même après le message \texttt{Done} du benchmark), conformément au diagnostic détaillé dans \texttt{docs/report/sections/A15\_boundary.md}.

Il est important de préciser que ce \texttt{Done} est imprimé par \texttt{test\_omp} et signifie uniquement que le calcul applicatif est terminé ; il ne garantit pas la fin correcte de toute l'exécution gem5, puisque l'état \texttt{DONE} n'est validé que si le processus gem5 se termine avec \texttt{exit=0}.

La mitigation \texttt{--omp-active-wait} (réduction de l'usage de la voie \texttt{futex}/mutex) a été déterminante pour stabiliser les cas en largeur 4, notamment à partir de \texttt{threads=16} et au-delà ; sans cette mitigation, plusieurs combinaisons échouaient.

\paragraph{Résultats numériques (extraits de \texttt{results/images/q9\_cycles.csv})}
\begin{center}
\scriptsize
\begin{tabular}{c c c}
\hline
\textbf{Width} & \textbf{Threads} & \textbf{Cycles} \\
\hline
2 & 2  & 1282419 \\
2 & 4  & 768565  \\
2 & 8  & 515053  \\
2 & 16 & 391465  \\
2 & 32 & 341535  \\
4 & 2  & 801594  \\
4 & 4  & 520290  \\
4 & 8  & 381832  \\
4 & 16 & 318392  \\
4 & 32 & 302483  \\
8 & 2  & 785008  \\
8 & 4  & 510238  \\
8 & 8  & 376396  \\
8 & 16 & 315224  \\
8 & 32 & 299612  \\
\hline
\end{tabular}
\captionof{table}{Cycles d'exécution obtenus pour Q9 (size=64).}
\label{tab:q9-cycles}
\end{center}

\paragraph{Visualisation 3D}
\begin{center}
\includegraphics[width=0.98\linewidth]{../../results/images/A15/q9_cycles_3d.png}
\captionof{figure}{Graphe 3D des cycles (X=threads, Y=voies/o3-width, Z=cycles).}
\label{fig:q9-cycles-3d}
\end{center}


\paragraph{Cycles de référence à 1 thread (extraits de \texttt{results/images/q9\_speedup.csv})}
\begin{center}
\scriptsize
\begin{tabular}{c c}
\hline
\textbf{Width} & \textbf{Cycles (threads=1)} \\
\hline
2 & 2308481 \\
4 & 1365568 \\
8 & 1334530 \\
\hline
\end{tabular}
\captionof{table}{Cycles de référence utilisés pour le calcul du speedup.}
\label{tab:q9-cycles-t1}
\end{center}

\paragraph{Calcul du speedup et résultats}
Le speedup est calculé, pour chaque largeur \(w\), par rapport au cas \texttt{threads=1} de la même largeur :
\[
S(w,t) = \frac{C_{w,1}}{C_{w,t}}
\]
où \(C_{w,t}\) est le nombre de cycles de la configuration \((w,t)\). Dans notre cas, les valeurs obtenues sont celles de \texttt{results/images/q9\_speedup.csv} :

\begin{center}
\scriptsize
\begin{tabular}{c c c c}
\hline
\textbf{Threads} & \textbf{Speedup (w=2)} & \textbf{Speedup (w=4)} & \textbf{Speedup (w=8)} \\
\hline
1  & 1.000 & 1.000 & 1.000 \\
2  & 1.800 & 1.704 & 1.700 \\
4  & 3.004 & 2.625 & 2.616 \\
8  & 4.482 & 3.576 & 3.546 \\
16 & 5.897 & 4.289 & 4.234 \\
32 & 6.759 & 4.515 & 4.454 \\
\hline
\end{tabular}
\captionof{table}{Speedup obtenu pour Q9 (size=64), calculé à partir de \texttt{q9\_speedup.csv}.}
\label{tab:q9-speedup}
\end{center}

\paragraph{IPC maximal par configuration}
Pour traiter la question sur l'IPC, nous avons créé le script \texttt{scripts/extract\_q9\_ipc.py}. Ce script lit \texttt{state.tsv} et les \texttt{stats.txt}, calcule :
\[
IPC(w,t)=\frac{\texttt{sim\_insts}(w,t)}{\texttt{cycles}(w,t)}
\]
et écrit les résultats dans :
\begin{itemize}
  \item \texttt{results/images/A15/q9\_ipc.csv},
  \item \texttt{results/images/A15/q9\_ipc\_max.csv}.
\end{itemize}

\begin{center}
\scriptsize
\begin{tabular}{c c c c}
\hline
\textbf{Width} & \textbf{Threads au max} & \textbf{IPC max}\\
\hline
2 & 32 & 14.696 \\
4 & 32 & 19.221 \\
8 & 32 & 23.301 \\
\hline
\end{tabular}
\captionof{table}{IPC maximal pour chaque largeur (size=64).}
\label{tab:q9-ipc-max-width}
\end{center}

Le maximum global observé est \(\mathbf{IPC=23.301}\), obtenu pour \texttt{width=8} et 
\texttt{threads=32}.

\subsection{Q12. Analyse de Résultats}

\paragraph{Limites d'exécution en gem5 (SE) : nombre de threads et stabilité}
Nous avions prévu d'explorer jusqu'à 64 threads, mais en pratique une limite nette de stabilité est apparue avec gem5-stable (mode SE) : au-delà d'un certain niveau de concurrence (dès \texttt{threads=40} dans nos essais), gem5 termine en \texttt{SIGSEGV} (\texttt{exit=139}), parfois \emph{après} que le benchmark ait affiché \texttt{Done} (le calcul applicatif est fini, mais la simulation n'est pas finalisée correctement). Cette limite est attribuée à une instabilité du simulateur plutôt qu'à un bug fonctionnel de \texttt{test\_omp} (voir le diagnostic expérimental dans \texttt{docs/report/sections/A15\_boundary.md}).

Un point important est que, dans ce TP, \texttt{nthreads = ncores} est imposé en mode SE (librairie pthreads en développement) : dans notre flot, \texttt{threads} est couplé à \texttt{--num-cpus} côté gem5 (cf. sujet du TP \texttt{docs/consigne.pdf}). Autrement dit, augmenter \texttt{threads} augmente aussi le nombre de cœurs simulés, ce qui amplifie la pression sur la hiérarchie mémoire, la cohérence, et la synchronisation.

\paragraph{Instabilité \texttt{futex}/mutex et mitigation \texttt{--omp-active-wait} (width $\ge 4$)}
En plus de la limite « trop de threads », nous avons observé des échecs plus tôt pour des largeurs superscalaires plus élevées (à partir de \texttt{width=4}) quand le nombre de threads augmente. La cause la plus probable est liée au chemin de synchronisation OpenMP/libgomp en Linux : lors des barrières/verrous, libgomp utilise \texttt{futex} (*fast userspace mutex*) pour endormir/réveiller des threads sans consommer de CPU. Or, en gem5 SE (notamment sur des versions anciennes), la prise en charge de certains cas \texttt{futex} peut être incomplète/instable, et la fréquence accrue des synchronisations à forte concurrence augmente la probabilité de déclencher ce problème.

La mitigation utilisée est \texttt{--omp-active-wait} (décrite dans \texttt{scripts/A15/A15\_commands.md}) qui injecte \texttt{OMP\_WAIT\_POLICY=ACTIVE} et un \texttt{GOMP\_SPINCOUNT} très élevé : les threads attendent davantage en \emph{spinning} en espace utilisateur, ce qui réduit les blocages/réveils via \texttt{futex}. Concrètement, cela a permis de faire passer des combinaisons qui échouaient auparavant (par exemple \texttt{width=4} avec plusieurs threads), même si, à très forte concurrence, une instabilité résiduelle peut encore persister (voir \texttt{docs/report/sections/A15\_boundary.md}).

\paragraph{Pourquoi les cycles diminuent quand on augmente les threads (TLP/CMP)}
La Table~\ref{tab:q9-cycles} montre une tendance monotone : à largeur fixe, plus le nombre de threads est grand, plus le nombre de cycles diminue. Cela s'explique principalement par le parallélisme au niveau des threads (TLP) sur une architecture CMP : chaque thread OpenMP exécute une partie du travail (multiplication de matrices) sur un cœur distinct, et le temps total est gouverné par le cœur le plus long (métrique \(\max(\texttt{system.cpu*.numCycles})\)). En augmentant \texttt{threads}, on réduit la quantité de travail par cœur et on augmente le parallélisme global, donc le nombre de cycles d'exécution diminue.

Le gain reste cependant sous-linéaire (cf. Table~\ref{tab:q9-speedup}) à cause (i) des portions sérielles incompressibles (création/fin des threads, initialisations), (ii) des surcoûts de synchronisation (barrières OpenMP), et (iii) des effets de hiérarchie mémoire/cohérence quand beaucoup de cœurs accèdent aux mêmes structures de données (bus/mémoire partagés).

\paragraph{Pourquoi les cycles diminuent quand on augmente la largeur superscalaire (ILP/OoO)}
À nombre de threads donné, on observe aussi une baisse des cycles quand \texttt{width} augmente
 (Table~\ref{tab:q9-cycles}). Ici, \texttt{width} correspond au degré de traitement 
 superscalaire côté cœur (dans nos scripts, c'est l'\texttt{issueWidth} du modèle o3). 
 Une largeur plus grande permet d'émettre davantage d'instructions par cycle quand le 
 code présente suffisamment d'indépendances (ILP), et l'exécution out-of-order contribue
 à mieux cacher des latences (par exemple en chevauchant calculs et accès mémoire). Cela
  correspond au positionnement “hautes performances” du Cortex-A15, conçu pour exploiter
   agressivement l'ILP.

La différence entre \texttt{width=4} et \texttt{width=8} devient toutefois plus faible à 
forte concurrence : une partie des cycles est alors contrainte par la synchronisation et la 
mémoire partagée, et non plus uniquement par la largeur d'émission d'instructions (rendements
 décroissants).

\paragraph{Lecture du graphe 3D}
La Figure~\ref{fig:q9-cycles-3d} illustre la même tendance sous forme de surface : on observe une « colline » pour \texttt{threads} faibles et \texttt{width} faible (beaucoup de cycles), puis une descente progressive quand on augmente l'un et/ou l'autre paramètre. Le « vallon » de cycles minimaux correspond à la zone de plus forte exploitation conjointe du parallélisme TLP (plus de cœurs) et de l'ILP (cœurs plus larges), dans la limite des surcoûts mémoire/synchronisation.

\paragraph{Speedup : pourquoi le maximum est à \texttt{width=2} sans être le meilleur temps absolu}
Dans la Table~\ref{tab:q9-speedup}, le meilleur speedup est obtenu pour \texttt{width=2} et \texttt{threads=32} (\(S=6.759\)), supérieur à \texttt{width=4} (\(4.515\)) et \texttt{width=8} (\(4.454\)). Ce résultat s'explique d'abord par la définition du speedup : \(S(w,t)=C_{w,1}/C_{w,t}\). Or, le cas mono-thread à \texttt{width=2} est nettement plus lent (\(C_{2,1}\) est bien plus grand que \(C_{4,1}\) et \(C_{8,1}\), Table~\ref{tab:q9-cycles-t1}), ce qui donne mécaniquement davantage de “marge” pour augmenter le ratio.

Ensuite, avec des cœurs plus larges (\texttt{width=4/8}), chaque cœur produit plus de requêtes et atteint plus vite un régime limité par des ressources partagées (mémoire, cohérence, barrières OpenMP). On « plafonne » donc plus tôt en speedup relatif, même si le temps absolu continue de baisser. En performance pure (cycles minimaux), la meilleure configuration observée est \texttt{width=8, threads=32} (\(299\,612\) cycles, Table~\ref{tab:q9-cycles}). Ainsi, le “gagnant” dépend du critère : \texttt{width=2} maximise le speedup relatif, tandis que \texttt{width=8} minimise les cycles (et correspond mieux à une configuration A15 performante).

\paragraph{IPC : maximum à \texttt{threads=32} et notion de « configuration la plus efficace »}
La Table~\ref{tab:q9-ipc-max-width} montre que, pour chaque largeur, l'IPC maximal est atteint à \texttt{threads=32}, et que le maximum global est obtenu pour \texttt{width=8, threads=32} avec \(IPC=23.301\). Dans notre extraction, l'IPC est calculé comme \(\texttt{sim\_insts}/\texttt{cycles}\) avec \(\texttt{cycles}=\max(\texttt{system.cpu*.numCycles})\) : c'est donc un \emph{IPC global} (débit d'instructions agrégé sur la durée critique), qui augmente naturellement quand on ajoute des cœurs/threads et quand les cœurs sont plus capables de retirer des instructions.

Dans le cadre de ce TP et de cette métrique, \texttt{width=8, threads=32} est bien la configuration la plus “efficace” au sens \emph{débit par cycle} et aussi la plus performante en cycles. En revanche, cela ne préjuge pas de l'efficacité énergétique ou du coût matériel (non modélisés ici), et le fait que \texttt{threads=32} soit maximal reflète aussi notre limite opérationnelle de stabilité (au-delà, gem5 devient instable).


% --- Section 5: Configuration CMP la plus efficace ---
\section{Configuration CMP la plus efficace}

\textbf{Q13. } D'après nos simulations, la configuration CMP avec deux cœurs Cortex-A7 (arm\_detailed) atteint un IPC d'environ \textbf{0.38} par cœur. Bien que cette valeur soit inférieure à celle d'un processeur Out-of-Order comme le A15, l'efficacité globale pour la multiplication de matrices est supérieure en termes de débit (throughput). 

En comparant avec le TP4, nous observons que les configurations ne sont pas strictement équivalentes en raison de l'hétérogénéité du matériel utilisé : les mesures du TP4 ont été effectuées sur une \textbf{machine locale (Intel i7-8750H)}, tandis que le TP5 a été exécuté sur le \textbf{serveur de l'école (Xeon E5-2640 v2)}. Cette différence de plateforme, notamment au niveau de la bande passante mémoire et de la gestion des caches, influe sur les résultats. 

Le compromis optimal reste une architecture hétérogène (type \textit{big.LITTLE}) : un cœur "Performance" pour les tâches séquentielles lourdes (comme \textbf{Dijkstra} qui nécessite de masquer les latences) et plusieurs cœurs "Efficacité" pour maximiser le parallélisme de calcul dans des tâches comme \textbf{MatMul}.

% --- Section 6: Facultatif ---
\subsection{Facultatif}
% Q14: Réponse sur la spécification et le Speedup Supra-linéaire
Le phénomène de \textbf{speedup supra-linéaire} observé s'explique par une synergie entre le parallélisme et la localité des données. Dans le TP4, une matrice 64x64 sur un seul cœur génère un \textit{miss rate} élevé ($\approx 30\%$). Dans notre simulation actuelle (TP5), les résultats du fichier \texttt{stats.txt} montrent seulement \textbf{3 454 misses} pour le CPU0 et \textbf{3 285} pour le CPU1, ce qui est extrêmement faible. En partitionnant la matrice, les données "tiennent" désormais dans les caches L1 privés. L'accélération dépasse donc le facteur $N$ car elle combine l'exécution parallèle et la suppression quasi-totale des cycles d'attente vers la RAM. Pour spécifier une architecture, il faut donc dimensionner le nombre de cœurs afin que le \textit{working set} par fil d'exécution ne dépasse jamais la capacité du cache L1/L2.

\input{sections/pt7_conclusion_extra.tex}



\begin{thebibliography}{00}

%TODO Delete this and add the real references of the project
\bibitem{TP5}
TP5, ``Analyse de performances de configurations de
microprocesseurs multicoeurs pour des applications parallèles,'' document de travaux pratiques du cours.

\bibitem{opencv_feature_description}
O. Hammami, \emph{Introduction \`a l'Architecture des Microprocesseurs}, ENSTA ParisTech, 828 Bvd des Mar\'echaux 91762 Palaiseau cedex, \url{https://www.ensta-paristech.fr}, ENSTA PARIS - Cours ES201, Ann\'ee 2022.

\end{thebibliography}

\end{document}
