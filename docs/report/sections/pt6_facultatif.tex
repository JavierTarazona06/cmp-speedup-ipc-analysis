% --- Section 6: Facultatif ---
\subsection{Facultatif}
% Q14: Réponse sur la spécification et le Speedup Supra-linéaire
Le phénomène de \textbf{speedup supra-linéaire} observé s'explique par une synergie entre le parallélisme et la localité des données. Dans le TP4, une matrice 64x64 sur un seul cœur génère un \textit{miss rate} élevé ($\approx 30\%$). Dans notre simulation actuelle (TP5), les résultats du fichier \texttt{stats.txt} montrent seulement \textbf{3 454 misses} pour le CPU0 et \textbf{3 285} pour le CPU1, ce qui est extrêmement faible. En partitionnant la matrice, les données "tiennent" désormais dans les caches L1 privés. L'accélération dépasse donc le facteur $N$ car elle combine l'exécution parallèle et la suppression quasi-totale des cycles d'attente vers la RAM. Pour spécifier une architecture, il faut donc dimensionner le nombre de cœurs afin que le \textit{working set} par fil d'exécution ne dépasse jamais la capacité du cache L1/L2.