\section{Paramètres de l'architecture multicoeurs}

\subsection{Q2 --- Paramètres par défaut d'un CPU OoO (BaseO3CPU)}

\paragraph{Remarque.}
Le fichier \texttt{O3CPU.py} sélectionne la variante OoO selon l'ISA.
Les \textbf{paramètres par défaut} se trouvent dans \texttt{src/cpu/o3/BaseO3CPU.py}.

\paragraph{Paramètres OoO retenus (valeurs par défaut).}
Nous reportons ci-dessous des paramètres structurants pour un cœur superscalaire out-of-order
(largeurs du pipeline et taille des structures de renommage/exécution), ainsi que leur interprétation.

\begin{table}[htbp]
\centering
\scriptsize
\caption{Paramètres OoO par défaut (extraits de \texttt{BaseO3CPU.py}) et rôle.}
\label{tab:q2_o3_params}
\begin{tabular}{|l|c|p{4.8cm}|}
\hline
\textbf{Paramètre} & \textbf{Défaut} & \textbf{Rôle / effet attendu} \\
\hline
\texttt{fetchWidth} & 8 & Max instructions fetch/cycle : alimente le pipeline ; utile si le front-end est limitant. \\
\hline
\texttt{decodeWidth} & 8 & Max decode/cycle : limite la capacité de décodage avant renommage/dispatch. \\
\hline
\texttt{issueWidth} & 8 & Max instructions émises/cycle : augmente l'ILP exploitable si dépendances faibles. \\
\hline
\texttt{commitWidth} & 8 & Max instructions retirées/cycle : borne finale du débit d'instructions (retirement). \\
\hline
\texttt{numROBEntries} & 192 & Taille du ROB : fenêtre OoO plus large $\Rightarrow$ meilleure tolérance aux latences (mémoire/branches). \\
\hline
\end{tabular}
\end{table}

\paragraph{Lecture.}
Ces valeurs indiquent un cœur OoO « large » (largeurs à 8) avec une fenêtre OoO conséquente (ROB=192).
Dans le cadre du TP, augmenter la largeur (via \texttt{o3-width}) revient à contraindre/décontraindre ces largeurs effectives,
ce qui agit sur la capacité à exploiter l'ILP à l'intérieur de chaque thread.

\subsection{Q3 --- Valeurs par défaut des caches (Options.py)}

\paragraph{Valeurs par défaut.}
D'après \texttt{configs/common/Options.py}, les paramètres par défaut sont :
\begin{itemize}
  \item \(L1D = 64\ \mathrm{KiB}\), associativité \(2\)-ways ;
  \item \(L1I = 32\ \mathrm{KiB}\), associativité \(2\)-ways ;
  \item \(L2 = 2\ \mathrm{MiB}\), associativité \(8\)-ways ;
  \item taille de ligne = \(64\ \mathrm{B}\).
\end{itemize}

\begin{table}[htbp]
\centering
\scriptsize
\caption{Paramètres de caches par défaut (Options.py).}
\label{tab:q3_cache_defaults}
\begin{tabular}{|l|c|c|}
\hline
\textbf{Niveau} & \textbf{Taille} & \textbf{Associativité} \\
\hline
L1D & 64KiB & 2 \\
\hline
L1I & 32KiB & 2 \\
\hline
L2  & 2MiB  & 8 \\
\hline
Cache line & 64B & -- \\
\hline
\end{tabular}
\end{table}

\paragraph{Commentaire.}
Ces valeurs conditionnent le taux de miss et la pression mémoire.
En multicœur, une augmentation de \(ncores\) augmente le volume de requêtes concurrentes, ce qui rend plus visibles
la contention sur le bus/mémoire et les surcoûts de cohérence.
